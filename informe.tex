\documentclass[11pt]{article}
\usepackage[spanish]{babel}
\usepackage{amsmath, amssymb, bm}
\usepackage[a4paper,margin=2.5cm]{geometry}
\usepackage{graphicx}
\usepackage{hyperref}
\usepackage{float}


\newcommand{\ii}{\mathrm{i}}

\title{Apuntes Quantum}
\author{
    Lukas Wolff}
\date{}

\begin{document}
\maketitle

\section{Qubit}

Un qubit se puede definir como:

\begin{equation}
    |\psi\rangle = \alpha |0\rangle + \beta |1\rangle
\end{equation}

Donde se debe cumplir por probabilidad que:

\begin{equation}
    |\alpha|^2 + |\beta|^2 = 1
\end{equation}

Ahora bien, se puede tener una superposicion de 2 o mas qubits, por ejemplo:

\begin{equation}
    |\psi\rangle = \alpha |00\rangle + \beta |01\rangle + \gamma |10\rangle + \delta |11\rangle
\end{equation}

Por lo tanto, el espacio lineal de n qubits es:

\begin{equation}
    |\psi\rangle = \sum_{}^{} \alpha_i |i\rangle
\end{equation}

\subsection{Angulos del qubit}

Un qubit se puede representar en una esfera de radio 1, conocida como esfera de Bloch. Para esto, se definen dos angulos $\theta$ y $\phi$:

\begin{equation}
    |\psi\rangle = \cos\left(\frac{\theta}{2}\right) |0\rangle + e^{\ii \phi} \sin\left(\frac{\theta}{2}\right) |1\rangle
\end{equation}

Donde, recordar que se un numero imaginario $a + bi$, se puede definiri como:

\begin{equation}
    a + bi = re^{\ii \phi} \quad \text{con} \quad r = \sqrt{a^2 + b^2} \quad \text{y} \quad \phi = \tan^{-1}\left(\frac{b}{a}\right)
\end{equation}


\subsection{Bases}

Las bases son conjuntos de qubits ortogonales entre si, que permiten representar cualquier qubit como una combinacion lineal de los qubits de la base. Algunas bases comunes son:

\begin{equation}
    |0\rangle = \begin{pmatrix} 1 \\ 0 \end{pmatrix}, \quad |1\rangle = \begin{pmatrix} 0 \\ 1 \end{pmatrix}
\end{equation}

\begin{equation}
    |+\rangle = \frac{1}{\sqrt{2}} (|0\rangle + |1\rangle) = \frac{1}{\sqrt{2}} \begin{pmatrix} 1 \\ 1 \end{pmatrix}, \quad |-\rangle = \frac{1}{\sqrt{2}} (|0\rangle - |1\rangle) = \frac{1}{\sqrt{2}} \begin{pmatrix} 1 \\ -1 \end{pmatrix}
\end{equation}

\begin{equation}
    |+y\rangle = \frac{1}{\sqrt{2}} (|0\rangle + \ii |1\rangle) = \frac{1}{\sqrt{2}} \begin{pmatrix} 1 \\ \ii \end{pmatrix}, \quad |-y\rangle = \frac{1}{\sqrt{2}} (|0\rangle - \ii |1\rangle) = \frac{1}{\sqrt{2}} \begin{pmatrix} 1 \\ -\ii \end{pmatrix}
\end{equation}


\section{Dual}

Sea un qubit base:

\begin{equation}
    |\psi\rangle = \begin{pmatrix} \alpha \\ \beta \end{pmatrix} = \alpha |0\rangle + \beta |1\rangle
\end{equation}

El dual de este qubit es:

\begin{equation}
    \langle \psi | = \begin{pmatrix} \alpha^* & \beta^* \end{pmatrix} = \alpha^* \langle 0| + \beta^* \langle 1|
\end{equation}

Donde la notacion $^*$ denota el conjugado complejo, es decir, si $\alpha = a + bi$, entonces $\alpha^* = a - bi$. El espacio lineal de los duales es:

\begin{equation}
    \langle \psi | = \sum_{}^{} \alpha_i^* \langle i |
\end{equation}

\section{Producto interno escalar o bra-ket}

Sea el qubit:

\begin{equation}
    |\psi\rangle = \begin{pmatrix} v_1 \\ v_2 \end{pmatrix}
\end{equation}

Y el dual:

\begin{equation}
    \langle \phi | = \begin{pmatrix} w_1^* & w_2^* \end{pmatrix}
\end{equation}

El producto interno escalar entre un qubit y un dual es:

\begin{equation}
    \langle \phi | \psi \rangle = \sum_{i} w_i^* v_i = w_1^* v_1 + w_2^* v_2
\end{equation}

Donde se cumple que:

\begin{equation}
    \langle \phi | \psi \rangle = \langle \psi | \phi \rangle^*
\end{equation}

Si dos vectores son ortogonales, se cumple que:

\begin{equation}
    \langle \phi | \psi \rangle = 0
\end{equation}

Y si es un vector consigo mismo, se cumple que:

\begin{equation}
    \langle \psi | \psi \rangle = 1
\end{equation}

\section{Producto externo o ket-bra}

Sea el qubit:

\begin{equation}
    |\psi\rangle = \begin{pmatrix} v_1 \\ v_2 \end{pmatrix}
\end{equation}

Y el dual:

\begin{equation}
    \langle \phi | = \begin{pmatrix} w_1^* & w_2^* \end{pmatrix}
\end{equation}

El producto externo entre un qubit y un dual es:

\begin{equation}
    |\psi\rangle \langle \phi | = \begin{pmatrix} v_1 w_1^* & v_1 w_2^* \\ v_2 w_1^* & v_2 w_2^* \end{pmatrix}
\end{equation}

Lo cual es la matriz de proyeccion del qubit $|\psi\rangle$ sobre el dual $\langle \phi |$.

Luego, para cualquier base se cumple que:

\begin{equation}
    I = \sum_{i} |i\rangle \langle i |
\end{equation}

Donde $I$ es el operador identidad, por ejemplo, para la base computacional:

\begin{equation}
    I = |0\rangle \langle 0| + |1\rangle \langle 1| = \begin{pmatrix} 1 & 0 \\ 0 & 0 \end{pmatrix} + \begin{pmatrix} 0 & 0 \\ 0 & 1 \end{pmatrix} = \begin{pmatrix} 1 & 0 \\ 0 & 1 \end{pmatrix}
\end{equation}

\section{Operadores}

Los operadores son matrices que actuan sobre los qubits, generando una rotacion del estado, o una transformacion del mismo. Algunos operadores basicos son:

\begin{itemize}
    \item Operador identidad:
    \begin{equation}
        I = \begin{pmatrix} 1 & 0 \\ 0 & 1 \end{pmatrix}
    \end{equation}
    
    \item Operador X (NOT):
    \begin{equation}
        X = \begin{pmatrix} 0 & 1 \\ 1 & 0 \end{pmatrix}
    \end{equation}
    
    \item Operador Y:
    \begin{equation}
        Y = \begin{pmatrix} 0 & -\ii \\ \ii & 0 \end{pmatrix}
    \end{equation}
    
    \item Operador Z:
    \begin{equation}
        Z = \begin{pmatrix} 1 & 0 \\ 0 & -1 \end{pmatrix}
    \end{equation}
    
    \item Operador Hadamard:
    \begin{equation}
        H = \frac{1}{\sqrt{2}} \begin{pmatrix} 1 & 1 \\ 1 & -1 \end{pmatrix}
    \end{equation}
\end{itemize}

\subsection{Operador NOT de una base cualquiera}

Para calcular el operador NOT de una base cualquiera, se debe conocer los vectores de la base. Sea la base:

\begin{equation}
    |u\rangle = \begin{pmatrix} a \\ b \end{pmatrix}, \quad |v\rangle = \begin{pmatrix} c \\ d \end{pmatrix}
\end{equation}

El operador NOT en esta base se define como:

\begin{equation}
    X_{uv} = |u\rangle \langle v| + |v\rangle \langle u| = \begin{pmatrix} a \\ b \end{pmatrix} \begin{pmatrix} c^* & d^* \end{pmatrix} + \begin{pmatrix} c \\ d \end{pmatrix} \begin{pmatrix} a^* & b^* \end{pmatrix}
\end{equation}

\subsection{Operador unitario}

Un operador unitario es aquel que cumple que:

\begin{equation}
    U^\dagger U = U U^\dagger = I
\end{equation}

Donde $U^\dagger$ es el conjugado transpuesto de $U$. Los operadores unitarios son importantes en la mecanica cuantica, ya que representan las transformaciones que pueden sufrir los estados cuanticos sin perder informacion. Se cumple que:

\begin{equation}
    \langle \psi | U^\dagger U | \psi \rangle = \langle \psi | I | \psi \rangle = \langle \psi | \psi \rangle = 1
\end{equation}

\subsection{Operaciones}

La operacion de un operador sobre un qubit se representa como:

\begin{equation}
    |\psi\rangle \rightarrow U = U |\psi\rangle
\end{equation}
\end{document}